% 中文摘要
\begin{abstract}
随着近几年中国汽车保有量的持续增加,智能交通系统以及自动驾驶技术俨然已成为十分
热门的研究领域。而机动车车牌识别技术作为这两项技术的基础和关键,有着十分重要的研
究意义和应用价值。本文主要对车牌识别系统所涉及的车牌检测、车牌定位、车牌字符分割
和车牌识别四个子系统展开研究,并运用最新的深度学习技术实现一个简单的车牌识别系统。
全文工作将分为以下几点:

第一,本文首先阐述本课题的研究背景及意义,并简要介绍深度学习技术相较于传统技术的
异同;

第二,本文尝试使用Faster R-CNN技术来进行车牌检测,来克服传统方法严重依
赖手工特征和手工规则、鲁棒性差等缺点;

第三,本文尝试使用CNN 回归车牌顶点坐标的方法以实现车牌定位;

第四,车牌字符分割问题可以看做对车牌图片中的文字进行
提取的问题。本文尝试使用Class-specific Extremal Region的方法进行车牌字符分割;

第五,在图像识别领域,目前最理想的方法当属卷积神经网络(简称CNN),它有着更好的
识别准确率和更强的鲁棒性。因此本系统采用CNN 的方法进行车牌识别。

最后,本文总结了深度学习技术及其在车牌识别这一计算机视觉任务上的应用前景,并提出
对未来的展望。
\end{abstract}
\keywords{车牌检测, 车牌定位, 车牌字符分割, 车牌识别, 深度学习}

% 英文摘要
\begin{enabstract}
With the continued increase in vehicle ownership in China in recent years,
intelligent transportation system and autopilot technology has became a very
popular research field. As the foundation and key of these two technologies,
the vehicle license plate recognition technology is of great importance in both
research and application. This paper focuses on four
essential subsystem in the vehicle license plate recognition system including
detection, localization, segmentation and recognition. We implement a simple vehicle
license plate recognition systems based on the up to date deep learning
technologies. We summarize the whole work as follows:

First, we illustrates the certain necessity of the research and implementation
of this vehicle license plate recognition system by describing the background and
significance of the research.

Second, This paper will
present an approach of vehicle license plate detection based on Faster R-CNN
technology, in order to overcome the shortcomings of traditional
methods such as the dependency of hand-made features as well as the
poor robustness.

Third, in this paper, we try to localize vehicle license plate by regressing the
vertex positions using CNN.

Fourth, The vehicle
license plate segmentation problem can be viewed as the problem of detecting
characters from license plate pictures. In this paper, we try to segment license
into character by using Class-specific Extremal Region technology.

Fifth, in the field of image recognition, there is no doubt that CNN is the
state-of-art approach currently, it has a higher accuracy and a better
robustness. So we adopt the CNN for license plate recognition in our final
system.

Sixth, through the study of the technologies above, this paper implements a
simple license plate recognition system. In our recognition system, we use RCNN
for detection, CNN regression for location, CSER for character segmentation and
CNN for the final recognition, respectively.
\end{enabstract}
\enkeywords{vehicle license plate detection, vehicle license plate location,
  vehicle license character segmentation, vehicle license plate recognition,
  deep learning}